\documentclass[11pt]{article}
\usepackage[margin=2.5cm]{geometry}
\usepackage{hyperref}
\usepackage{enumitem}

\title{FDML-Core Project: Conversational Tutor Briefing}
\author{Shiv Saransh Thakur}
\date{\today}

\begin{document}
\maketitle

\section{Big picture in one sentence}

At a high level, this project is a small publishing pipeline for folk dances.
Instead of keeping dances in scattered notes or PDFs, I define a simple XML
language called FDML (Folk Dance Markup Language). From that XML, the
toolchain can validate the data and automatically produce caller cards,
PDFs and a small searchable web site.

\section{What FDML actually is (without code)}

FDML is just a structured way of writing down a dance. Each FDML file has two
main parts: meta and body.

\begin{itemize}
  \item \textbf{Meta}: title of the dance, the name, the musical meter
        (for example 3/4 or 4/4), tempo in beats per minute and an optional
        author.
  \item \textbf{Body}: the actual material of the dance: sections of text,
        figures and sequences.
\end{itemize}

Informally:

\begin{itemize}
  \item A \textbf{figure} is one choreographic chunk, for example a basic step
        or a promenade.
  \item A \textbf{step} is a single action inside a figure: who moves,
        what they do and how many beats it takes.
  \item A \textbf{sequence} is a list of figures that describes the order
        of the dance.
\end{itemize}

There are rules to keep things tidy:

\begin{itemize}
  \item Every figure has an id and at least one step.
  \item Step beat counts must be positive.
  \item When a sequence refers to a figure, that figure must actually exist
        in the same file.
  \item Titles and emails are checked for basic sanity.
\end{itemize}

\section{How the toolchain works conceptually}

The Java project \texttt{fdml-core} is essentially a command line wrapper
around three ideas: validate, render and index.

\subsection{Validation in three layers}

Validation happens in three layers, each one adding more meaning:

\begin{enumerate}
  \item \textbf{Schema layer}: XML Schema (XSD) checks that the file is
        structurally valid FDML: elements appear where they should, attributes
        have the right types and so on.
  \item \textbf{Rule layer}: Schematron adds richer rules: titles are non
        empty, beats are positive, references are valid, emails look sensible.
  \item \textbf{Lint layer}: a linter looks at musical meter and total beats
        in each figure and warns when they do not line up cleanly.
\end{enumerate}

From the user point of view there are simple commands:

\begin{itemize}
  \item \texttt{fdml validate}: schema only.
  \item \texttt{fdml validate-sch}: Schematron only.
  \item \texttt{fdml validate-all}: both together.
  \item \texttt{fdml lint}: only meter consistency warnings.
  \item \texttt{fdml doctor}: run everything and give a combined report.
\end{itemize}

\subsection{Rendering and PDFs}

Rendering is done by XSLT stylesheets that transform FDML into HTML.
Conceptually:

\begin{itemize}
  \item input: an FDML XML file,
  \item transform: an XSLT file that lays it out as a caller card with figures,
        steps and approximate bars,
  \item output: HTML suitable for the web or for printing.
\end{itemize}

There is also a path that turns the HTML into a PDF using a headless HTML to
PDF converter. In conversation I can simply say that once a dance is written
in FDML, I can produce a printable caller card and a PDF automatically.

\subsection{Indexing and search}

The project also builds a small JSON index over all the FDML files. For each
dance it records the file path, title, author email (if present) and any
section ids. A static search page written in plain HTML and JavaScript loads
this JSON and lets the user type a query. Matching dances are shown as cards.
There is no database or server: the whole thing is a static site that can be
hosted on GitHub Pages.

\section{Current status (what is working now)}

For the tutor meeting, this is the important status summary:

\begin{itemize}
  \item The FDML 1.0 schema and Schematron rules are implemented and stable.
  \item The linter for meter and beat consistency is implemented.
  \item The command line tool supports validation, linting, rendering,
        PDF export and index generation.
  \item A small example corpus of dances is encoded in FDML.
  \item There is a Makefile and scripts that build the HTML cards, the JSON
        index and the static site (index page plus search page).
  \item The project builds and runs on Java 17, with automated tests for
        rendering and indexing.
  \item Continuous integration runs these checks on every push.
  \item A human readable implementation specification file (FDML-SPEC) exists
        in the repository.
\end{itemize}

In one sentence: the core pipeline is in place. I can write dances in FDML,
validate them, generate cards and build a small searchable site, and there
are tests and CI around that.

\section{What is left to do}

There are several follow up tasks and polish items:

\begin{itemize}
  \item Richer index and search: include formation, total bars and figure
        information in the index, and move towards more faceted search
        (for example filter by formation).
  \item Stronger Schematron and lint rules: better messages and additional
        checks around figures, sections and sequences.
  \item Corpus expansion: encode more real dances to show that the format
        generalises beyond a small set of examples.
  \item Documentation for authors: write a short guide on how to go from a
        notebook description of a dance to a clean FDML file, plus a tutorial
        style usage guide for the command line.
  \item Evaluation and reflection: discuss what the design handles well
        (consistency, publishing, validation) and where it is limited
        (very complex structures, integration with music and so on).
\end{itemize}

\section{Phrases I can reuse in the meeting}

Some ready made sentences I can use if the tutor asks me to explain things:

\begin{itemize}
  \item On the goal: ``I am treating folk dances as first class data. FDML is
        a simple markup for describing dances, and the project is a toolchain
        that can validate and publish them.''
  \item On the pipeline: ``Given an FDML file I can run a single command line
        tool to check it structurally and semantically, then generate a
        printable caller card and have it appear in a small searchable site.''
  \item On the status: ``The core schema, validation, rendering and static
        site are implemented. Now I am focusing on richer search, a larger
        corpus and better author facing documentation.''
  \item If pressed on implementation details: ``Under the hood it is Java 17
        plus XSLT and Schematron, wrapped in a simple fdml command line so
        that an end user does not need to know the internals.''
\end{itemize}

\end{document}

