\documentclass[11pt]{article}
\usepackage[margin=2.5cm]{geometry}
\usepackage{hyperref}
\usepackage{enumitem}

\title{FDML-Core Project: Tutor Briefing (with Plain-Language Explanations)}
\author{Shiv Saransh Thakur}
\date{\today}

\begin{document}
\maketitle

\section{What my project is, in one sentence}

My project is a \textbf{small publishing pipeline for folk dances}.

\medskip
Instead of keeping dances in scattered notes or random documents, I use a simple
text format called \textbf{FDML} (Folk Dance Markup Language) to describe
dances in a structured way, and then I have tools that:

\begin{itemize}
  \item check that the description makes sense, and
  \item automatically produce caller cards, PDFs and a small searchable web site.
\end{itemize}

If I need to summarise quickly to my tutor, I can say:

\begin{quote}
``I treat folk dances as structured data: I write them in a simple markup
language (FDML), then I run a tool that validates them and turns them into
printable cards and a small searchable site.''
\end{quote}

\section{Key concepts explained in plain language}

\subsection*{XML}

\textbf{XML} is just a way of writing structured text with tags like
\texttt{<meta>} or \texttt{<figure>}. It is similar in spirit to HTML but used
for data rather than web pages.

In my project, an FDML file is an XML file that follows some rules.

\subsection*{FDML}

\textbf{FDML (Folk Dance Markup Language)} is \emph{my} specific XML format for
folk dances. Conceptually each FDML file has:

\begin{itemize}
  \item a \textbf{meta} section (title, dance name, meter, tempo, author), and
  \item a \textbf{body} section (sections of text, figures and sequences).
\end{itemize}

Plain language:

\begin{itemize}
  \item A \textbf{figure} is one chunk of choreography, such as a basic step or promenade.
  \item A \textbf{step} is one action inside a figure: who moves, what they do,
        how many beats it takes.
  \item A \textbf{sequence} is a list of figures in order, describing the whole dance.
\end{itemize}

\subsection*{Schema (XSD)}

A \textbf{schema} is like a template or form that says:

\begin{quote}
``These elements are allowed, these attributes are allowed, and this is the
structure they must appear in.''
\end{quote}

In my project, the XML Schema (XSD) defines what a valid FDML file \emph{looks}
like structurally.

\subsection*{Schematron rules}

\textbf{Schematron} is a way to express extra rules that are more like
\emph{logic} than structure. For example:

\begin{itemize}
  \item ``Every figure must have at least one step.''
  \item ``If there is an email address, it must look roughly like an email.''
  \item ``If a sequence refers to figure \texttt{f-basic}, that figure must exist.''
\end{itemize}

So:

\begin{itemize}
  \item The schema says \emph{which tags and attributes are allowed}.
  \item Schematron says \emph{what combinations make semantic sense}.
\end{itemize}

\subsection*{Linter}

A \textbf{linter} is a tool that looks for suspicious or inconsistent patterns
and gives warnings rather than hard errors.

In my project, the linter looks at:

\begin{itemize}
  \item the meter (for example 3/4), and
  \item the total beats in each figure,
\end{itemize}

and warns if they do not line up nicely. For example, if a figure in 3/4 time
has a total number of beats that does not divide well into bars, the linter
flags it as an \emph{off\_meter} warning.

\subsection*{CLI (Command-Line Interface)}

A \textbf{CLI} is just a tool you run from the terminal, for example:

\begin{quote}
\texttt{fdml validate corpus/valid/example-01.fdml.xml}
\end{quote}

In conversation I can say: \emph{``The project provides a small command-line
tool called \texttt{fdml} that I run to validate and render dances.''}

\subsection*{XSLT}

\textbf{XSLT} is a template language that says how to turn one XML document into
another format, such as HTML.

In my project, XSLT stylesheets describe how to lay out an FDML dance as a
caller card: which information to show, how to show beats and bars, and how to
format figures and sequences.

\subsection*{JSON index and static site}

\textbf{JSON} is a very simple data format (lists and key--value objects) that
JavaScript can read easily in a browser.

The project builds a JSON file that lists all the dances and some metadata (for
example file path, title, sections). Then a tiny search page (pure HTML +
JavaScript) loads that JSON and lets the user type a query to filter dances.

It is a \textbf{static site}: there is no backend server or database. Everything
is just files that can be hosted on GitHub Pages or any static web host.

\section{What the toolchain does, step by step}

Conceptually, the toolchain works like this:

\begin{enumerate}
  \item I or another author write a dance in FDML (that is, in my XML format).
  \item I run the \texttt{fdml} command line tool to:
    \begin{itemize}
      \item check structure (schema),
      \item check rules (Schematron),
      \item and get any meter warnings (linter).
    \end{itemize}
  \item When the file looks good, I run commands to:
    \begin{itemize}
      \item render HTML caller cards,
      \item generate a JSON index,
      \item and build the static web pages.
    \end{itemize}
  \item Optionally, I run a command that turns the rendered HTML into a PDF
        suitable for printing or sharing.
\end{enumerate}

In conversation, I can describe this as:

\begin{quote}
``Write the dance in FDML, run the tool to validate it, and then generate
cards, PDFs and a small searchable web site from the same source.''
\end{quote}

\section{Current status (what is working now)}

For the tutor meeting, this is the important status summary in non-technical language:

\begin{itemize}
  \item The FDML 1.0 format is defined and implemented as an XML schema.
  \item Extra rules (Schematron) are implemented to catch common mistakes (for
        example missing steps, bad references, empty titles).
  \item A linter exists that looks at meter and beat counts and warns about
        figures that do not fit nicely.
  \item The \texttt{fdml} CLI can:
    \begin{itemize}
      \item validate FDML files,
      \item run the linter,
      \item render HTML cards,
      \item export to PDF,
      \item build a JSON index over the corpus.
    \end{itemize}
  \item There is a small example corpus of dances encoded in FDML.
  \item There is a Makefile and scripts that build:
    \begin{itemize}
      \item the example HTML cards,
      \item the JSON index,
      \item and the static web pages (index page and search page).
    \end{itemize}
  \item The project builds and runs on Java~17, with automated tests that check
        rendering and indexing.
  \item Continuous integration on GitHub runs these checks on every push.
  \item A separate \textbf{FDML-SPEC} document exists in the repo describing
        the format in more detail.
\end{itemize}

If my tutor asks \emph{``So is it actually working?''} I can say:

\begin{quote}
``Yes. I can take an FDML file, run the CLI, and it will validate it, render a
caller card, and include it in a small searchable site. The example corpus
already goes through this pipeline, and tests and CI check that I do not break
it.''
\end{quote}

\section{What is left to do (roadmap)}

Planned next steps and polish, described in human terms:

\begin{itemize}
  \item \textbf{Richer index and search:}
    \begin{itemize}
      \item Add more metadata to the index, such as formation and total bars.
      \item Move towards a more ``faceted'' search where you can filter by
            formation or other properties, not just by text.
    \end{itemize}
  \item \textbf{Stronger rules and linting:}
    \begin{itemize}
      \item Improve error messages so they are more user friendly.
      \item Add more checks for unusual or inconsistent structures.
    \end{itemize}
  \item \textbf{Bigger corpus:}
    \begin{itemize}
      \item Encode more real dances so that the system covers a wider variety
            of formations and choreographies.
    \end{itemize}
  \item \textbf{Author documentation:}
    \begin{itemize}
      \item Write a short ``How to write FDML'' guide that takes someone from a
            dance in their notebook to a clean FDML file.
      \item Write a tutorial-style explanation of the \texttt{fdml} CLI
            (how to validate, how to generate cards, how to build the site).
    \end{itemize}
  \item \textbf{Evaluation and reflection:}
    \begin{itemize}
      \item Reflect on what the design supports well (consistency, publishing,
            validation).
      \item Reflect on what it does not yet support (very complex dances,
            integration with music or audio, and so on).
    \end{itemize}
\end{itemize}

If my tutor asks \emph{``What would you do if you had more time?''} I can talk
about these points.

\section{Phrases I can reuse in the meeting}

Some ready-made sentences I can use if I am put on the spot:

\begin{itemize}
  \item On the goal:
    \begin{quote}
    ``I am treating folk dances as first-class data. FDML is a small markup
    for describing dances, and the project is a toolchain that can validate
    and publish them.''
    \end{quote}
  \item On the pipeline:
    \begin{quote}
    ``Given an FDML file I can run a single command line tool to check it
    structurally and semantically, then generate a printable caller card and
    have it appear in a small searchable site.''
    \end{quote}
  \item On the current status:
    \begin{quote}
    ``The core schema, validation, rendering and static site are implemented.
    Now I am focusing on richer search, a larger corpus and better
    author-facing documentation.''
    \end{quote}
  \item If pressed on implementation details:
    \begin{quote}
    ``Under the hood it is Java~17 plus XSLT and Schematron, wrapped in a
    simple \texttt{fdml} command line so that an end user does not need to
    know the internals.''
    \end{quote}
\end{itemize}

If I forget a detail in the meeting, I can say I need to double-check in the
repository, but this briefing is designed so that I can talk about the project
confidently at a conceptual level.

\end{document}

